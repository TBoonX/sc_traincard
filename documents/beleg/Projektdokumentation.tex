\documentclass[a4paper,12pt]{scrartcl} 

%% \usepackage[latin1]{inputenc}
%%apt-get install texlive-lang-german kile kile-l10n  aspell-de  texlive-fonts-extra
%%damit ngerman keine Probleme mehr macht !!
\usepackage[utf8]{inputenc} 
\usepackage[T1]{fontenc}
\usepackage[ngerman]{babel}

\usepackage{setspace}

\setcounter{tocdepth}{3}				%Schatelungstiefe Inhaltsverz.
\usepackage[utf8]{inputenc}			%deutsche Umlaute
\usepackage{german, ngerman}
\usepackage[ngerman]{babel}			%Rechtschreibprüfung
\usepackage{color,listings} 			%Quellcode Highlighting, bindet das
\usepackage{float}					%% GRAFIKPOSITION MITTELS [H] ERWZINGEN
%Paket Listings ein
\usepackage{listings}
\usepackage{color}
\usepackage{textcomp}
\usepackage[T1]{fontenc}				%srccode
\usepackage[scaled]{beramono}		%srccode
\usepackage{longtable}				%mehrseitige tabellen
\usepackage[tableposition=b]{caption}
\usepackage[pdftex, pdftoolbar=false, hyperfootnotes=false, bookmarks,
bookmarksopen, bookmarksnumbered, bookmarksopenlevel=2, pdfpagelabels=true,
pdfstartpage=3, pdfstartview=FitH,]{hyperref} %Verlinkungen
\usepackage{array}					%farbige Tabellen
\usepackage[table]{xcolor} 			%farbige Tabellen
\usepackage{graphicx}				% \includegraphics bnoetigt dies

\usepackage{fancyhdr, graphicx}		%% Logo auf Titelseite
\renewcommand{\headrulewidth}{0pt}
\fancyhead[L]{}
\fancyhead[R]{
  \includegraphics[width=52mm]{./images/htwk.png}
}

%%%% mathemathische Formeln zentrieren und vom Text absetzen mittels \[ E = mc^2 \] anstatt $ E = mc^2 $ %%%%
\usepackage{amsmath}
\usepackage{amsthm}
\usepackage{amsbsy}
\usepackage{amssymb}
%%%%%%%%%%%%%%%%%%%%%%%%%%%%%%%%%%%%%%%%%%%%%%%%%%%%%%%%%%%%%%%

\definecolor{Navy}{rgb}{0,0,0.5}
\definecolor{Gray}{gray}{0.5}
\definecolor{dunkelgrau}{rgb}{0.8,0.8,0.8}
\definecolor{hellgrau}{rgb}{0.95,0.95,0.95}
\definecolor{hellgrau2}{rgb}{0.93,0.93,0.93}

\hypersetup{
	colorlinks=true, 			% false: boxed links; true: colored links
	linkcolor=Navy,          		% color of internal links
	citecolor=Gray,        			% color of links to bibliography
	filecolor=magenta,      		% color of file links
	urlcolor=blue,           			% color of external links
	linkbordercolor={1 1 1}, 		% set to white
	citebordercolor={1 1 1} 		% set to white
}


%Einrückung eines neuen Absatzes
\setlength{\parindent}{0em}

%Definition der Ränder
\usepackage[paper=a4paper,left=30mm,right=30mm,top=30mm,bottom=30mm]{geometry}

%Abstand der Fussnoten
\deffootnote{1em}{1em}{\textsuperscript{\thefootnotemark\ }}

%Regeln, bis zu welcher Tiefe (section,subsection,subsubsection) Überschriften angezeigt werden sollen (Anzeige der Überschriften im Verzeichnis / Anzeige der Nummerierung)
%\setcounter{tocdepth}{3}
%\setcounter{secnumdepth}{3}

\fancypagestyle{htwkheader}
{
   \fancyhf{}	% clear all header and footer fields
  \fancyhead[RO]{
	\makebox[\textwidth]{	%% schiebe Logo nach aussen auf den Rand
		\rule{1				%% nach aussen schieben hoeherer Wert -> Logo weiter nach aussen
		  \textwidth}{0cm} %% nicht nach unten schieben = 0cm
			\includegraphics*[width=52mm]{./images/htwk.png}	%%Logo HTWK
	  }
  }
}



\begin{document}
 
%Beginn der Titelseite
\begin{titlepage}
\begin{small}
\vfill {HTWK Leipzig\\
Fachbereich IMN \\
Sommersemester 2014}
\end{small}
 
\begin{center}
\begin{Large}
\vfill {\textsf{\textbf{
Traincard\\
}}}
\end{Large}
Beleg im Smartcard Programmierung\\bei Prof. Dr. rer. nat. Uwe Petermann
\end{center}
 
\begin{small}

\vfill
Kurt Junghanns, B.Sc.\\
Marcel Kirbst, B.Sc. \\
Michael Reher, B.Sc.\\
\\
\today
\end{small}
 
\end{titlepage}
%Ende der Titelseite
 
%Inhaltsverzeichnis (aktualisiert sich erst nach dem zweiten Setzen)
\tableofcontents
%Abbildungsverzeichnis und Tabellenverzeichnis auf einer Seite
\clearpage
\listoffigures
\listoftables
\thispagestyle{empty}
 
\clearpage
\onehalfspacing
 
\pagestyle{plain}
 
\section{Einleitung}
\label{sec:0}
Diese Arbeit befasst sich mit der Vorstellung der Belegarbeit im Fach Smartcard Programmierung mit dem Thema ``Traincard''. 
Ziel des Belegs ist die prototypische Implementierung eines Trainingssystems auf Basis JCOP-fähiger Smartcards. 
\\
\\
Im ersten Kapitel \nameref{sec:1} wird der theoretische Hintergrund und der Anwendungsfall für die prototypische Implementierung beschrieben, dieser Unterteilt sich in den \nameref{subsec:1.1} und den \nameref{subsec:1.2}. \\
Im Kapitel \nameref{sec:2} werden die grundlegenden Technologien (Smartcard, JCOP und APDU) und deren Standards vorgestellt. \\
Kapitel \nameref{sec:3} erläutert detailliert die Kommunikation mit der Smartcard, erläutert die verwendete Datenstruktur und beschreibt den On-Card und den Off-Card Teil der Implementierung. \\
Den Abschluss bildet das Kapitel \nameref{sec:4}, welches einen kurzen Überblick über die Vorgehensweise und deren Ergebnisse bietet.

\clearpage
\section{Anwendungsfall}
\label{sec:1}
In Fitnessstudios werden heute noch bei der Neuanmeldung Trainingspläne auf Papier ausgegeben. Der Sportler führt den Trainingsplan über die Trainingsperiode ständig bei sich und verzeichnet den Trainingsfortschritt in diesem Dokument.
Im Verlauf der Trainingsperiode kann der Trainer beurteilen, wie effektiv das Training beim Trainierenden ist.\\
\\
Die Verwendung von Trainingspläne auf Papier hat jedoch einige Nachteile. Beispielsweise wird das Dokument vom Sportler manchmal vergessen, die betreffenden Trainingsfortschritte müssen also nachgetragen werden. 
Weiterhin sind Traingspläne in Papierform nur bedingt resistent gegen Abnutzung und verschleißen mit fortdauernder Verwendung.
Um den genannten Nachteilen zu begegnen soll im Rahmen dieser Belegarbeit der Einsatz von Trainingsplänen
auf Basis so genannter Smartcards abgebildet werden. Smartcards können sehr leicht in einer Brieftasche mitgeführt werden und sind weniger anfällig für Abnutzung.
Weiterhin bieten sie weitere Vorteile wie automatische elektronische und anonyme Datenerfassung zu den Sportlern, die Möglichkeit die Daten extern zu sichern und bei Bedarf auf eine beliebige Smartcard zu exportieren.

\subsection{Anwendungsfall aus Sicht des Sportlers}
\label{subsec:1.1}

Sportler können mit Traincard folgende Aktionen durchführen:
\begin{itemize}
\item gesamten Trainingsplan einsehen
\item Tagesplan einsehen
\item tagesbezogenen Trainingsfortschritt speichern
\item Trainingsfortschritt jeder einzelnen Übung einsehen
\end{itemize}

Aus Sicht des Sportlers bietet die Verwendung der Smartcard einige der folgenden Vorteile:
\begin{itemize}
\item leichte Transportierbarkeit
\item robuster als Trainingspläne aus Papier
\item übersichtliche Darstellung des eigenen Trainingsfortschrittes
\end{itemize}


\subsection{Anwendungsfall aus Sicht des Trainers}
\label{subsec:1.2}
Traincard bietet Trainern folgende Anwendungsmöglichkeiten:
\begin{itemize}
\item gesamten Trainingsplan erstellen
\item gesamten Trainingsplan einsehen
\item Tagesplan einsehen
\item Trainingsfortschritt jeder einzelnen Übung einsehbar
\end{itemize}

Trainer profitieren beim Einsatz von Smartcards unter anderem von folgenden Vorteilen:
\begin{itemize}
\item verständliche Auswertung der vorgegebenen Trainingspläne
\item Trainingsplan nur durch Trainer veränderbar
\item Smartcard mit Eintrittskarte in das Fitnessstudio koppelbar\footnote{Fitnessstudios wie \glqq Fitness Nr. 1\grqq \ auf der Karl-Liebknecht Straße in Leipzig verwenden Karten mit Magnetstreifen als Zugangskarte}
\end{itemize}

\clearpage
\section{Grundlagen}
\label{sec:2}
In diesem Kapitel wird auf die Grundlagen der in diesem Beleg verwendeten Technologien eingegangen.

\subsection{Grundlagen Smartcard}
\label{subsec:2.1}
Die in diesem Beleg verwendete Smartcard basiert auf der Java Card Technologie. Die Java Card Technologie bietet eine 
Teilmenge der Java Programmiersprache, sowie eine hinsichtlich der Anforderungen an Smartcards optimierte Laufzeitumgebung.

Die Verwendung von Java-basierten Smartcards bietet einige Vorteile, von denen nachfolgend eine Auswahl beispielhaft genannt sei: \cite{jcopdoc}

\begin{description}
\item[plattformunabhängig:] Java Card Applets, die der Java Card API entsprechen, lassen sich plattformunabhängig und herstellerübergreifend nutzen.
\item[Multiapplikationsfähig:] es können mehrere Applikationen gleichzeitig auf einer Smartcard ausgeführt werden
\item[hohe Flexibilität:] die Verwendung der objektorientierten Programmiersprache Java erlaubt die Erstellung komplexer Anwendungen für die Smartcard.  
\item[Post-Aktualisierbarkeit:] die Möglichkeit, nachträglich Code auf der Smartcard zu modifizieren und auszutauschen erhöht die Flexibilität weiter
\item[Standardkonformität:] die Java Smartcards entsprechen dem ISO7816 Standard \cite{iso7816}
\end{description}

Eine Java Smartcard besteht im Wesentlichen aus den Bestandteilen Kommunikationsschnittstelle, Speicher und einem Prozessor zur Durchführung von Berechnungen.
Bei Verwendung der Smartcard wird diese in ein Lesegerät eingelegt. Das Lesegerät wird in einschlägiger Literatur auch als Card Acceptance Device, abgekürzt CAD, bezeichnet.
Der Speicher auf Java Smartcards besteht aus zwei Typen, RAM und EEPROM. Der RAM-Speicher ist flüchtig und kann beliebig oft beschrieben werden. Der EEPROM-Speicher ist nichtflüchtig und kann nur endlich oft beschrieben werden, je nach Hersteller und Modell bis zu 100.000 mal pro Speicherzelle.  

\subsection{JCOP}
\label{subsec:2.2}
Die Java Card Open Plattform, abgekürzt JCOP, ist ein Smartcard Betriebssystem das iniitial von IBM entwickelt wurde, inzwischen jedoch von NXP Semiconductors betreut wird. Der Titel leitet sich von den Namen der zu Grunde liegenden Spezifikationen für Java Card und Global Platform (früher bekannt unter Open Plattform) ab.\\


Bei der Softwareentwicklung kann auf reale sowie auf simulierte JCOP-Karten zurückgegriffen werden. Der Zugriff auf eine reale JCOP-Karte erfolgt mittels eines Terminal-Kartenlesers sowie spezieller Treiber. Da während der Erarbeitung des Belegs keine Hardware zur Verfügung stand, wurde die Implementierung an einer simulierten JCOP-Karte getestet. Auf die Entwicklungsumgebung wird im Kapitel \nameref{sec:3} näher eingegangen.


 


\subsection{APDU}
\label{subsec:2.3}

Die Kommunikation zwischen dem Kartenlesegerät und der Smartcard erfolgt reaktiv und paketweise. Reaktiv bedeutet, dass die Smartcard keine Kommunikation initiiert sondern nur auf Anfragen vom Lesegerät reagiert. Der bei der Kommunikation zwischen Smartcard und Lesegerät verwendete Kommunikationsmechanismus wird als Application Protocol Data Units, abgekürzt APDU, bezeichnet. Spezifiziert ist dies ebenfalls in ISO7816 Teil 4.\cite{iso7816}
\\

\begin{figure}[htb]
\begin{center}
 \includegraphics[width=1\hsize]{./images/apdu.png}
\end{center}
\caption[APDU-Kommunikation schematisch dargestellt, Quelle: Autoren]{\label{apdu}APDU-Kommunikation schematisch dargestellt, Quelle: Autoren}
\end{figure}

Die Kommunikation wird über ein so genanntes command APDU Paket initiiert, welches aus den folgenden Feldern besteht:


\begin{description}
\item[CLA:] class,gibt die Klasse an, spezifiziert ob es sich um ein ISO7816-4 konformes Kommando handelt
\item[INS:] gibt die Instruktion an
\item[P1:] zusätzlicher Parameter
\item[P2:] zusätzlicher Parameter
\end{description} 

Weiterhin können je nach Kommandotyp noch die folgenden, optionalen Felder an das command APDU Paket angehängt werden:

\begin{description}
\item[Lc:] Length, gibt die Länge der Kommandodaten
\item[Data:] gibt die Kommandodaten an
\item[Le:] gibt die Länge der erwarteten Antwort an
\end{description}

Als Antwort erhält das Lesegerät von der Smartcard ein so genanntes response APDU Paket. Dieses Antwortpaket kann ein Datenfeld enthalten, dies ist jedoch nicht obligatorisch. Das response APU Paket ist wie folgt aufgebaut.

\begin{description}
\item[Data:] optionales Datenfeld
\item[Sw1:] Statusword, erstes Byte
\item[Sw2:] Statusword, zweites Byte
\end{description}







\clearpage
\section{Umsetzung}
\label{sec:3}

Dieses Kapitel legt die Umsetzung dar.

\subsection{Kommunikation}
\label{subsec:3.0}

Die Anwendung besteht aus einem On-Card und einem Off-Card Teil.\\
Entsprechend der JCOP Umgebung, ist der On-Card Teil durch ein Applet realisiert, welches auf der Smartcard installiert und gestartet wird.
Diese Smartcard kann eine physische oder eine emulierte zum Einsatz kommen.
Im Rahmen dieses Projektes wird die Smartcard per JCOP Eclipse Umgebung emuliert und verwendet.
Dabei ist in der gestarteten JCOP Shell der Befehl $/close$ auszuführen, wodurch die Smartcard für externe  Zugriffe auf der lokalen IP des Emulator-Rechners auf dem Port 8090 erreichbar ist.

Der Off-Card Teil ist ebenfalls in Java geschrieben und kommuniziert mit Hilfe des OpenCard Frameworks mit der Smartcard.
\\

Die Kommunikation zwischen On-Card und Off-Card auf verschiedenen Rechnern benötigt entsprechende Regeln der Firewalls der Rechner, lokale Ausführung beider Teile auf einem Rechner hingegen keine.

Der Inhalt der APDUs orientiert sich stark an der Norm ISO 7816-4. Der konkrete Aufbau ist in Abbildung \ref{myapdu} dargestellt.\\
Die Abkürzungen stehen dabei für folgendes:
\begin{description}
\item[CLA] ein Byte, welches immer mit dem Wert 0 belegt wird, da die APDU nicht vollständig ISO 7816-4 konform ist
\item[INS] ein Byte, welches die Instruktion angibt
\item[NOA] ein Byte, welches die Anzahl an gesamt zu übertragenden APDUs zur vollen Abarbeitung der Instruktion angibt
\item[LEN] ein Byte, welches die Anzahl der folgenden Daten-Bytes aufzeigt
\item[DATA] beliebige Anzahl von Bytes, welche die Daten enthalten\footnote{Anzahl der Daten-Bytes sollte LEN entsprechen}
\item[SW1] ein Byte, welches den vorderen Teil der Statusrückgabe darstellt
\item[SW2] ein Byte, welches den zweiten Teil der Statusrückgabe enthält
\end{description}

Die Statusrückgabe wird automatisch durch die JCOP Umgebung an die APDU angehangen. Die Interpretation der Statusrückgabe ist anhand ISO 7816-4 durchzuführen.

\begin{figure}[htb]
\begin{center}
 \includegraphics[width=1\hsize]{./images/myapdu.png}
\end{center}
\caption[Selbstdefinierter Dateninhalt der APDUs]{\label{myapdu}Selbstdefinierter Dateninhalt der APDUs}
\end{figure}

APDUs enthalten bis zu 256 Bytes, davon bis zu 252 Bytes für Daten.
Es sind ganze Trainingspläne und andere Objekte im Byte Format zwischen On-Card und Off-Card Teil zu übertragen. Dazu existiert pro Datenmodell in der dazugehörigen Klasse je eine Funktion zur Umwandlung von Instanz zu Byte Array und zur Erzeugung einer Instanz aus einem Byte Array. Diese Umwandlung  wurde eigens implementiert und orientiert sich am Aufbau von Ethernet Paketen. Ein Paket oder Byte Array besteht dabei immer aus einem Byte für einen eindeutigen Identifikator des Modells, zwei Bytes für die Anzahl der folgenden Bytes sowie Daten Bytes.
Eine Instanz des Modells MyDate mit den Attributen Jahr\footnote{Bei einem Jahr werden nur die letzten zwei Ziffern hexadezimal abgespeichert.}, Monat und Tag wird in folgendes Byte Repräsentation umgewandelt: $01$ $00$ $03$ $0e$ $07$ $0b$
\\

%verschlüsselung
Bei der Realisierung des Trainingssystemes werden keine Sicherheitsrelevanten Daten zwischen On-Card und Off-Card Teil übertragen. Auf Grund dessen wurde auf eine verschlüsselte Kommunikation mit asymmetrischer oder symmetrischer Verschlüsselung verzichtet.
Einzig Rollen-spezifische Schreibvorgänge sind in beiden Card Teilen durch je ein Passwort geschützt.
Der Trainer und der Sportler besitzen jeweils ein Passwort, welches mit der Java Bibliothek java.security.MessageDigest und derdarin enthaltenen  Hashfunktion SHA-256 zu einem 32 Byte langem Wert gehasht.
Dieser Hash ist auf der Smartcard gespeichert und muss bei der Anmeldung zum anschließenden Vergleich des Trainers oder Sportlers übertragen werden.
Der jeweilige Benutzer gibt in der grafischen Oberfläche sein Passwort im Klartext ein worauf die Programmlogik dieses hasht und es an die Smartcard überträgt.
Initial ist die Karte mit je einem Hash für Trainer und Sportler zu füllen. Im Zuge der Vereinfachung sind diese Werte bereits statisch gefüllt.
Um den Passworthash des Trainers zu setzen ist das in Listing \ref{lst:setpswds} dargestellte Kommando an die Smartcard zu senden.

\begin{lstlisting}[language=java, captionpos=b, caption=Setzen des Passwortes eines Sportlers in Form eines Kommandos, label=lst:setpswds]
/send 00 05 01 21 02 
\end{lstlisting}

\subsection{On-Card Teil}
\label{subsec:3.2}
Der On-Card Teil wird durch das Applet $Traincard$ repräsentiert.
\\


Im Klassendiagramm in der Abbildung \ref{diaoncard} ist das Applet mit dessen Klassenvariablen und Methoden dargestellt.
Das Applet befindet sich im Package\\htwk.smartcard.traincard und muss auf der verwendeten Smartcard installiert und gestartet werden.
\\

Entsprechend der Oberklasse javacard.framework.Applet, existieren Methoden, welche die Installation, Ausführung und Deselektion behandeln.
Die wichtigste Methode ist dabei process, welche ankommende APDUs erhält, diese auswertet, die entsprechende Klassenmethode aufruft und deren Antwort-Bytes in Form einer Response APDU zurücksendet.

\begin{figure}[htb]
\begin{center}
 \includegraphics[width=.7\hsize]{./images/Klassendiagramm_oncard.png}
\end{center}
\caption[Klassendiagramm On-Card Teil]{\label{diaoncard}Klassendiagramm On-Card Teil}
\end{figure}


%das applet empfängt anfragen, wertet diese aus und sendet Daten oder ein erfolgsbyte im datenbereich zurück
%Daten werden dabei immer als byte arrays geschrieben und gelesen

%passwörter sind gehashed abgelegt

Das Instruktionsbyte wird mit den statischen Konstanten verglichen und anhand eines Treffers die entsprechende Methode aufgerufen. Jede dieser Methoden erhält den Puffer der APDU und gibt ein Byte Array für den Puffer der Response APDU zurück.
Der Aufbau der Command und Response APDU ist in der Abbildung \ref{myapdu} dargestellt.
Die darin dargestellten Data Bytes sind entsprechend der jeweiligen Instruktion gefüllt.\\
$00$ $02$ $01$ $01$ $01$ als APDU liest beispielsweise den Trainingsplan und sendet ihn zurück. Ist der Trainingsplan größer als MAXRESPONSEDATALENGTH Byte, muss mit $00$ $02$ $01$ $01$ $02$ eine weitere APDU mit den nächsten Bytes des Trainingsplanes angefordert werden.
\\

Lokale Byte Arrays der Methoden werden stets flüchtig mit der Methode makeTransientByteArray der Klasse javacard.framework.JCSystem angelegt.
Der Vergleich und das Kopieren von Byte Arrays erfolgt mit den Methoden arrayCompare und arrayCopy der Klasse javacard.framework.Util.
Dadurch wird der EEPROM der Smartcard nicht durch Schreibvorgänge belastet und die Ausführungszeit verringert sich.
\\

Der Trainingsplan und der Fortschritt wird in dynamischer Größe auf der Smartcard gespeichert und bei jedem Schreibvorgang neu erzeugt.

\subsection{Off-Card Teil}
\label{subsec:3.3}

% der offcard Teil besteht aus der graphischen Oberfläche, der Logik zur Steuerung der Oberfäche und der Logik zur Kommunikation mit der Smartcard
%screenshots mit erklärungen was gemacht wird + grob technisches
%passwörter werden gehashed
%Packagediagramm

\clearpage
\section{Zusammenfassung und Ausblick}
\label{sec:4}
Der beschriebene Anwendungsfall wurde im Rahmen dieses Belegs analysiert und eine Lösung implementiert. 

\clearpage
\section{Quellenverzeichnis}
\label{sec:6}
\renewcommand\refname{Quellenverzeichnis}
\begin{thebibliography}{999}

\bibitem{uwe}Prof. Dr. rer. nat. Petermann, Uwe: {\sl Lehrmaterial zur Veranstaltung Smartcard-Programmierung im SS2014 an der HTWK-Leipzig.}\\
\url{https://bildungsportal.sachsen.de/opal/auth/RepositoryEntry/437649412/CourseNode/87246817112798}\\
intern abrufbar am 10.Juli 2014

\bibitem{jcopdoc}Sun Microsystems, Inc.:  {\sl Java Card Applet Developer's Guide}\\
\url{http://www.oracle.com/technetwork/java/javacard/downloads/index.html}\\
abrufbar am 01.Juli 2014

\bibitem{iso7816}International Organization for Standardization: {\sl ISO 7816 - Identification cards -- Integrated circuit cards}
kostenpflichtig abrufbar unter \url{http://www.iso.org/}

\end{thebibliography}
\end{document}