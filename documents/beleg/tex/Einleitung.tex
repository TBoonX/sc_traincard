\section{Einleitung}
\label{sec:0}
Diese Arbeit befasst sich mit der Vorstellung der Belegarbeit im Fach Smartcard Programmierung mit dem Thema ``Traincard''. 
Ziel des Belegs ist die prototypische Implementierung eines Trainingssystems auf Basis JCOP-fähiger Smartcards. 
\\
\\
Im ersten Kapitel \nameref{sec:1} wird der Use-Case für die prototypische Implementierung beschrieben, sowohl unter \nameref{subsec:1.1} aus Sicht des Trainierenden, wie auch unter \nameref{subsec:1.2} aus Sicht des Trainers.
%In Kapitel \nameref{sec:2} wird die Protokollfamilie IPv6 vorgestellt. Welche Probleme gelöst werden, wenn anstelle von IPv4, IPv6 verwendet wird, wird im Kapitel \nameref{sec:3} erläutert. Im Kapitel \nameref{sec:4} wird anhand eines Praxisbeispiels erläutert wie sich IPv6-Konnektivität an einem Rechnersystem kofigurieren lässt. Abschließend erfolgt im Kapitel \nameref{sec:5} eine Zusammenfassung der Arbeit.


%%Auch wenn klar sein muss, dass es im Umfang eines solchen Beleges nicht ann\"ahrend abschlie{\ss}end m\"oglich ist das Thema IPv6 ersch\"opfend vorzustellen, so hofft der Autor dem geneigten Leser doch einen \"Uberblick \"uber die Funktionen und Vorteile von IPv6 bieten zu k\"onnen.