\clearpage
\section{Anwendungsfall}
\label{sec:1}
In Fitnessstudios werden heute noch bei der Neuanmeldung Trainingspläne auf Papier ausgegeben. Der Sportler führt den Trainingsplan über die Trainingsperiode ständig bei sich und verzeichnet den Trainingsfortschritt in diesem Dokument.
Im Verlauf der Trainingsperiode kann der Trainer beurteilen, wie effektiv das Training beim Trainierenden ist.\\
\\
Die Verwendung von Trainingspläne auf Papier hat jedoch einige Nachteile. Beispielsweise wird das Dokument vom Sportler manchmal vergessen, die betreffenden Trainingsfortschritte müssen also nachgetragen werden. 
Weiterhin sind Traingspläne in Papierform nur bedingt resistent gegen Abnutzung und verschleißen mit fortdauernder Verwendung.
Um den genannten Nachteilen zu begegnen soll im Rahmen dieser Belegarbeit der Einsatz von Trainingsplänen
auf Basis so genannter Smartcards abgebildet werden. Smartcards können sehr leicht in einer Brieftasche mitgeführt werden und sind weniger anfällig für Abnutzung.
Weiterhin bieten sie weitere Vorteile wie automatische elektronische und anonyme Datenerfassung zu den Sportlern, die Möglichkeit die Daten extern zu sichern und bei Bedarf auf eine beliebige Smartcard zu exportieren.

\subsection{Anwendungsfall aus Sicht des Sportlers}
\label{subsec:1.1}

Sportler können mit Traincard folgende Aktionen durchführen:
\begin{itemize}
\item gesamten Trainingsplan einsehen
\item Tagesplan einsehen
\item tagesbezogenen Trainingsfortschritt speichern
\item Trainingsfortschritt jeder einzelnen Übung einsehen
\end{itemize}

Aus Sicht des Sportlers bietet die Verwendung der Smartcard einige der folgenden Vorteile:
\begin{itemize}
\item leichte Transportierbarkeit
\item robuster als Trainingspläne aus Papier
\item übersichtliche Darstellung des eigenen Trainingsfortschrittes
\end{itemize}


\subsection{Anwendungsfall aus Sicht des Trainers}
\label{subsec:1.2}
Traincard bietet Trainern folgende Anwendungsmöglichkeiten:
\begin{itemize}
\item gesamten Trainingsplan erstellen
\item gesamten Trainingsplan einsehen
\item Tagesplan einsehen
\item Trainingsfortschritt jeder einzelnen Übung einsehbar
\end{itemize}

Trainer profitieren beim Einsatz von Smartcards unter anderem von folgenden Vorteilen:
\begin{itemize}
\item verständliche Auswertung der vorgegebenen Trainingspläne
\item Trainingsplan nur durch Trainer veränderbar
\item Smartcard mit Eintrittskarte in das Fitnessstudio koppelbar\footnote{Fitnessstudios wie \glqq Fitness Nr. 1\grqq \ auf der Karl-Liebknecht Straße in Leipzig verwenden Karten mit Magnetstreifen als Zugangskarte}
\end{itemize}