\clearpage
\section{Anwendungsfall}
\label{sec:1}
In Fitnessstudios werden heute noch bei der Neuanmeldung Trainingspläne auf Papier ausgegeben. Der Trainierende führt den Trainingsplan über die Trainingsperiode ständig bei sich und verzeichnet den Trainingsfortschritt in diesem Dokument.
Im Verlauf der Trainingsperiode kann der Trainer beurteilen, wie effektiv das Training beim Trainierenden ist.\\
\\
<<<<<<< HEAD
Die Verwendung altmodischer Trainingspläne auf Papier hat jedoch einige Nachteile. Beispielsweise wird das Dokument vom Trainierenden manchmal vergessen, die betreffenden Trainingsfortschritte müssen also nachgetragen werden. 
Weiterhin sind Traingspläne in Papierform nicht besondern resistent gegen Abnutzung und verschleißen mit fortdauernder Verwendung. Um den genannten Nachteilen zu begegnen soll im Rahmen dieser Belegarbeit der Einsatz von Trainingsplänen
auf Basis so genannter Smartcards abgebildet werden. Smartcards können sehr leicht in einer Brieftasche mitgeführt werden und sind weniger anfällig für Abnutzung. Eine Auflistung einiger Vorteile bei der Verwendung von Smartcards ist nachfolgen aufgeführt.
=======
Die Verwendung von Trainingspläne auf Papier hat jedoch einige Nachteile. Beispielsweise wird das Dokument vom Trainierenden manchmal vergessen, die betreffenden Trainingsfortschritte müssen also nachgetragen werden. 
Weiterhin sind Traingspläne in Papierform nicht besondern resistent gegen Abnutzung und verschleißen mit fortdauernder Verwendung.
Um den genannten Nachteilen zu begegnen soll im Rahmen dieser Belegarbeit der Einsatz von Trainingsplänen
auf Basis so genannter Smartcards abgebildet werden. Smartcards können sehr leicht in einer Brieftasche mitgeführt werden und sind weniger anfällig für Abnutzung.
Weiterhin bieten sie weitere Vorteile wie automatische elektronische und anonyme Datenerfassung zu den Sportlern, die Möglichkeit die Daten extern zu sichern und bei Bedarf auf eine beliebige Smartcard zu exportieren.
% ... (Datenlogging für das Studio, Kopieren bzw. verteilter Zugriff auf die Daten ....) 

>>>>>>> 18d914365971a3816d843bf6c52e869f309cca67

\subsection{Anwendungsfall aus Sicht des Trainierenden}
\label{subsec:1.1}

Aus Sicht des Trainierenden bietet die Verwendung der Smartcard einige der folgenden Vorteile:
\begin{itemize}
\item leichte Transportierbarkeit
\item robuster als Trainingspläne aus Papier
\end{itemize}


\subsection{Anwendungsfall aus Sicht des Trainers}
\label{subsec:1.2}
Trainer profitieren beim Einsatz von Smartcards unter anderem von folgenden Vorteilen:
\begin{itemize}
\item leichte und lesbare Auswertung der vorgegebenen Trainingspläne
\item zusätzliche Metainformationen wie beispielsweise zu welchem Zeitpunkt welcher Datensatz geschrieben wurde
\item statistische Auswertungen lassen sich sehr leicht erstellen 
\item manipulationssicher
\item es fallen zusätzlich Metadaten an die ausgewertet werden können (z.B.: wann wurde die Karte an welchem Gerät benutzt)
\end{itemize} 
