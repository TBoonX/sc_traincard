\clearpage
\section{Zusammenfassung und Ausblick}
\label{sec:4}

Im Rahmen dieses Beleges sollte ein On-Card und ein Off-Card Teil erstellt werden, welcher ein Trainingssystem für einen Trainer und einen Sportler bereitstellt.
Dabei sollten die Elemente Trainingsplan, Trainingsfortschritt und Tagesplan im Rahmen der Benutzerrollen verfügbar und veränderbar sein.
\\
Dieses Ziel wurde erreicht. Der Programmcode wurde implementiert, die Programmteile sind funktionsfähig und eine entsprechende Dokumentation liegt vor.
\\
Die Anforderungen wurden zu Beginn definiert und stellten die Grundlage der folgende Modellierung dar.
Die Modelle und die Programmstruktur bildeten die Grundlage für die anschließende Programmierarbeit und die Erstellung der Dokumentation.
Weiterhin ermöglichte dieses Vorgehen eine gleichberechtigte Arbeitsteilung.
\\

\glqq Traincard\grqq \ ist mit JCOP fähigen Smartcards und einem Java fähigen Hostsystem für das Off-Card Programm einsetzbar.
Dadurch sind die in Kapitel \nameref{sec:1} beschriebenen Anforderungen erfüllt.
Durch den beschrieben Aufbau der APDUs und die Trennung von On- und Off-Card, besteht die Möglichkeit, einen der Teile auszutauschen ohne den anderen abändern zu müssen.
\\

Dieser Beleg ist auf github.com unter dem öffentlichen Projekt \footnote{\url{https://github.com/TBoonX/sc_traincard}} zu finden.